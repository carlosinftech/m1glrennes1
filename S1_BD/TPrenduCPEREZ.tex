\documentclass{article}

\usepackage[utf8]{inputenc} 
\usepackage[T1]{fontenc}
\usepackage{url} 
\usepackage[francais]{babel}
\usepackage{graphicx}
\DeclareGraphicsExtensions{.pdf,.png.jpeg}




% ------------------------
% DEBUT Modifier

\title{Rapport Travails Pratiques basse de donnees}
\author{Carlos Perez}

% FIN Modifier
% ------------------------





\begin{document}

\maketitle 
\tableofcontents
\date{03 decembre 2015}



% ------------------------
% DEBUT Modifier

\section{Interprétation du plan d’exécution}
\label{laSectionUne}

Cette section parle de l,evaluation des plans d'execution pour les requetes
du point 1 du TP.  \ref{laSection}. 

\subsection{1.1}
\label{laSousSectionUne}

\textbf{INDEX UNIQUE SCAN:}\verb!SELECT annee_naissance from artiste where id = 37;! ;
\paragraph{} Nombre de pages accedes en RAM 2.
\paragraph{ Plan d'execution}
\verb!----------------------------------------------------!
\verb!| Id  | Operation                   | Name         |!
\verb!----------------------------------------------------!
\verb!|   0 | SELECT STATEMENT            |              |!
\verb!|   1 |  TABLE ACCESS BY INDEX ROWID| ARTISTE      |!
\verb!|   2 |   INDEX UNIQUE SCAN         | SYS_C0098771 |!
\verb!----------------------------------------------------!;

\subsection{1.2}
\begin{itemize} 
\item \textbf{INDEX UNIQUE SCAN:} La basse de donnèes cherche la valeur dans l'index
et arrete quand la valeur a ete trouve parce que ce ne pas possible d'avoir plus 
d'un valeur.
\item \textbf{TABLE ACCESS INDEX BY ROWID:} On utilisse un index pour acceder directement
au ressources de la basse de données.
\end{itemize} 
 

\subsection{1.3}
Requete:
\verb!SELECT annee_naissance from artiste where id > -1;! 
\paragraph{} Nombre de pages accedes en RAM 7.

\paragraph{ Plan d'execution}

-------------------------------------
| Id  | Operation         | Name    |
-------------------------------------
|   0 | SELECT STATEMENT  |         |
|   1 |  TABLE ACCESS FULL| ARTISTE |
-------------------------------------

\subsection{1.4}

\begin{itemize} 
\item \textbf{TABLE ACCESS FULL:} La basse de donnèes lisse tous les registres
de la table. Dans ce cas le probleme c'est que la requete n'est pas selective,
en d'autres mots, la requete contient la majoritè des valeurs de la table. C'est 
pour cette raison que l'optimisseur a realis'e un scan complet de la table meme
si on utilise un indice dans la requete. 
\end{itemize} 

\subsection{1.5}
\paragraph{} Il me semble raisonnable la choix du methode parce qu'utiliser l'index
ne donnerai aucune avantage dans ce cas specifique. 


\subsection{1.6}Requete:

\verb!SELECT nom,prenom from artiste inner join (SELECT id_acteur from role inner join!   
\verb!(SELECT film.id AS idfilm from film inner join (SELECT id from artiste where nom = 'Tarantino')! 
\verb!realisateur on film.id_realisateur = realisateur.id) realisateur! 
\verb!on role.ID_FILM = realisateur.idfilm) filmes on artiste.id = filmes.id_acteur;! 



\verb!-----------------------------------------------------!
\verb!| Id  | Operation                    | Name         |!
\verb!-----------------------------------------------------!
\verb!|   0 | SELECT STATEMENT             |              |!
\verb!|   1 |  NESTED LOOPS                |              |!
\verb!|   2 |   NESTED LOOPS               |              |!
\verb!|   3 |    NESTED LOOPS              |              |!
\verb!|   4 |     HASH JOIN                |              |!
\verb!|   5 |      TABLE ACCESS FULL       | ARTISTE      |!
\verb!|   6 |      TABLE ACCESS FULL       | FILM         |!
\verb!|   7 |     INDEX RANGE SCAN         | SYS_C0098788 |!
\verb!|   8 |    INDEX UNIQUE SCAN         | SYS_C0098771 |!
\verb!|   9 |   TABLE ACCESS BY INDEX ROWID| ARTISTE      |!
\verb!-----------------------------------------------------!

\subsection{1.7}

\begin{itemize} 
\item \textbf{INDEX RANGE SCAN} La basse de donnèes cherche les valeurs dans un rangue
possible dans l'index. L'optimiseur utilise une structure de donnèes en arbre pour trouver 
\item \textbf{INDEX UNIQUE SCAN} La basse de donnèes cherche la valeur dans l'index
et arrete quand la valeur a ete trouve parce que ce ne pas possible d'avoir plus 
d'un valeur.
\item \textbf{HASH JOIN} L'optimiseur charge en memoire le resultat d'appliquer une
fonction de hash a la plus petite de deux basse de donnèes disponibles. Apres il compare
les valeurs donnees par la function contre les valeurs de la table plus grande.
Le resultat et la disminution des access innecessaires au disque.

\item \textbf{NESTED LOOP} Cette methode cree deux boucles un pour la table grande et
l'autre pour la table petite. Le boucle interne s'execute pour chaque file de la table
exterieure. 
\item \textbf{TABLE ACCESS INDEX BY ROWID:} On utilisse un index pour acceder directement
au ressources de la basse de données.
\end{itemize} 


\subsection{1.8}
\paragraph{ }La solution choisie est trop efficace parce qu'elle fait seulement
deux access complets dans une table et utilise des indices. L'access a la table artiste
etait necessaire car 


\section{Vues}


\subsection{2.1}
\label{laSousSectionDEuxUn}
Vue:
\verb!CREATE view artisterole as SELECT nom,prenom from artiste  inner join!
\verb!(Select id_acteur, count(*) AS nomfilmes from role group by id_acteur) rolefilm ON ! 
\verb!artiste.id = rolefilm.id_acteur AND rolefilm.nomfilmes > 3;! 

\subsection{2.2}
\label{laSousSectionDeuxDeux}
Requete:
\verb!SSELECT * FROM artisterole;! 



\section{PL/SQL}
Dans cette section on montre les declencheurs generees et la table genere

\subsection{3.1}
Requete:
\verb!create or replace TRIGGER trigger_artistes!
\verb!BEFORE INSERT!
\verb!   ON artiste!
\verb!  FOR EACH ROW!
   
\verb!DECLARE!

   
\verb!BEGIN!

\verb!   :new.annee_naissance := :new.annee_naissance*10;!
   
\verb!END;! 

\subsection{3.2}Requete:
\verb!CREATE TABLE nbupdates! 
\verb!( chaine VARCHAR2(50),! 
\verb!  nombre_updates INT! 
\verb!);! 

\subsection{3.3}

Declencheur:
\verb!create or replace TRIGGER trigger_nbupdates! 
\verb!AFTER INSERT! 
\verb!   ON artiste! 
\verb!   FOR EACH ROW! 
\verb!DECLARE! 
\verb!  vupd integer;! 
\verb!BEGIN! 

\verb!   SELECT nombre_updates INTO vupd FROM nbupdates where chaine ='ARTISTES';! 
   
\verb!   UPDATE nbupdates! 
\verb!   SET nombre_updates = vupd+1! 
\verb!   WHERE chaine ='ARTISTES';! 
\verb!   EXCEPTION! 
\verb!   WHEN NO_DATA_FOUND THEN! 
\verb!      dbms_output.put_line('La chaine ARTISTES n''existe pas dans la table nbupdates');! 
\verb!   WHEN OTHERS THEN! 
\verb!      raise_application_error(-20001,'Un erreur a ete trouvé - '||SQLCODE||' -ERROR- '||SQLERRM);! 

\verb!END;! 




\end{document}

